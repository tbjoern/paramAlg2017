%%%%%%%%%%%%%%%%%%%%%%%%%%%%%%%%%%%%%%%%%%%%%%%%%%%%%%%%%%%%%%%%%%%%%%
%% all the formatting stuff and packages

\documentclass[11pt]{article}

\parindent0em
\parskip.5em

\usepackage{xstring}

\usepackage{amsthm}
\usepackage[utf8]{inputenc}
\usepackage[ttscale=.85]{libertine}
\usepackage{libertinust1math}
% \usepackage[libertine,cmintegrals,cmbraces,vvarbb]{newtxmath}
\usepackage[T1]{fontenc}
\usepackage{microtype}
\usepackage{amsmath}
\usepackage{amssymb}
\usepackage{xspace}
\usepackage{ngerman}
\usepackage{graphicx}
\usepackage{lastpage}
\usepackage{ifthen}
\usepackage{fp}
\usepackage{hyperref}
\usepackage{icomma}
\usepackage{paralist}
\usepackage[ngerman,onelanguage,noend]{algorithm2e}
\DontPrintSemicolon

\makeatletter
\DeclareRobustCommand{\bfseries}{%
   \not@math@alphabet\bfseries\mathbf
   \fontseries\bfdefault\selectfont
   \boldmath
}
\makeatother

\usepackage[headsep=1cm]{geometry}

\usepackage{fancyhdr}

\fancypagestyle{plain}{%
  \renewcommand{\headrulewidth}{0pt}%
  \fancyhf{}
  \rhead{\includegraphics[width = 2.4cm]{fig/hpi_logo.pdf}}
  \lhead{\textbf{\sffamily Parametrisierte Algorithmen} \\
    \textbf{\sffamily Wintersemester 2017/2018} \\
    \sffamily Thomas Bläsius}
  \cfoot{\thepage}
  \rfoot{\ifthenelse{\thepage < \pageref{LastPage}}{\textit{bitte
        wenden}}{}}
}
\fancypagestyle{normal}{%
  \renewcommand{\headrulewidth}{0pt}%
  \fancyhf{}
  \cfoot{\thepage}
  \rfoot{\ifthenelse{\isodd{\thepage} \and \thepage <
      \pageref{LastPage}}{\textit{bitte wenden}}{}}
}

\usepackage{titlesec}

\titleformat{\section}%
[hang]%
{\Large\bfseries\sffamily}%
{Aufgabe \thesection:}%
{.5em}%
{}%
[]

\renewcommand{\thesubsection}{\alph{subsection}}
\usepackage{titlesec}
\titleformat{\subsection}%
[runin]%
{\bfseries\sffamily}%
{Teilaufgabe (\thesubsection)}%
{0pt}%
{}%
[]

\usepackage{titlesec}
\titleformat{\subsubsection}%
[hang]%
{\large\bfseries\sffamily}%
{\thesection}%
{.5em}%
{}%
[]

\titlespacing{\section}{0pt}{1.5ex}{.5ex}

\titlespacing{\subsection}{0pt}{.5ex}{.5em}

\titlespacing{\subsubsection}{0pt}{1ex}{.5ex}


%%%%%%%%%%%%%%%%%%%%%%%%%%%%%%%%%%%%%%%%%%%%%%%%%%%%%%%%%%%%%%%%%%%%%%
%% commands to use in the exercise sheet/solution

\newcommand{\sheet}[2]{ %
  \title{\textbf{\sffamily Übungsblatt #1}\\[-0.5ex]
    {\normalsize \sffamily Abgabe bis #2}}
  \date{}
  \maketitle
  \pagestyle{normal}
  \vspace{-2cm}
}

\newcommand{\solution}[2]{ %
  \title{\textbf{\sffamily Musterlösung zum Übungsblatt #1}\\[-0.5ex]
    {\normalsize \sffamily Erstellt von #2}}
  \date{}
  \maketitle
  \pagestyle{normal}
  \vspace{-2cm}
}

\newcommand{\exercise}[2][]{%
  \section{#2 \hfill {\normalsize#1}}%
}

\newcommand{\subexercise}{%
  \subsection{}%
}

\newcommand{\how}[1]{%
  \subsubsection*{Wie kommt man drauf?}%
}




\DeclareMathOperator{\vc}{vc}

\begin{document}

\solution{6}{Sören Tietböhl, Tobias Stengel}

\exercise{$d$-\textsc{Hitting Set}}
Sei eine $d$-\textsc{Hitting Set} Instanz gegeben. Zunächst kann ein Hitting Set $H$ der Größe $\leq k$ gefunden werden, indem man nur die ersten $k$ Mengen betrachtet und jeweils ein Element dieser $k$ Mengen zu $H$ hinzugefügt wird. Nun kann schrittweise ein neues Element $S_i$ zur Betrachtung hinzugenommen werden und davon ein Element zu $H$ hinzugefügt werden. Dann erhält man ein Hitting Set für $k+1$ Elemente, welches die Größe $k+1$ hat. Kann man dieses Hitting Set nun mittels eines Disjoint-Algorithmus auf die Größe $k$ reduzieren, kann man mittels wiederholtem Hinzufügen und Anwenden des Disjoint-Algorithmus die Instanz für alle $S_i$ Mengen ($1 \leq i \leq n$) lösen und erhält ein Hitting Set $H$ mit $|H| \leq k$.

Ziel des Disjoint-Algorithmus ist es, bei gegebener Instanz und gegebenem Hitting Set $H$ der Größe $k+1$ ein Hitting Set $H^\prime$ der Größe $k$ zu finden, ohne bestimmte Elemente aus $H$ zu benutzen. Dazu rät man zunächst eine Teilmenge von zu benutzenden Elementen aus $H$, man unterteilt $H$ damit in zu benutzende und nicht zu benutzende Elemente für $H^\prime$. Da $H$ ein Hitting Set ist, gibt es in jeder Menge $S_i$ ein Element aus $H$. Daher kann pro Menge $S_i$ wie folgt verfahren werden:

\begin{itemize}
\item Enthält $S_i$ ein Element aus $H$, welches als zu benutzend markiert ist, kann die Menge ignoriert werden, da sie bereits ein Element enthält, welches auch in $H^\prime$ vorkommt. $S_i$ wird dann also schon vom Hitting Set $H^\prime$ abgedeckt.

\item Enthält $S_i$ kein Element aus $H$, welches als zu benutzend markiert ist, so muss es ein Element enthalten, welches nicht zu $H^\prime$ hinzugefügt werden darf. Dieses Element ist damit irrelevant und kann aus $S_i$ entfernt werden. Galt vorher $|S_i| = d$, gilt damit jetzt $|S_i|=d-1$.

\end{itemize}

Das heißt alle Mengen $S_i$ sind entweder schon abgedeckt oder müssen mit einer um 1 verringerten Kardinalität betrachtet werden. Statt eines $d$-\textsc{Hitting Sets} wird also ein $d$-1-\textsc{Hitting Set} gesucht. 

Fügt man nun eine neue Menge $S_i$ hinzu, kann dieser Schritt erneut ausgeführt werden. Damit wird $d$ immer weiter reduziert, bis irgendwann $d=2$ gilt. Betrachtet man die Sets einer $2$-\textsc{Hitting Set}-Instanz als eine Kante und die beiden Elemente der Sets als Knoten, kann das $2$-\textsc{Hitting Set} auch als Vertex Cover dargestellt werden, da pro existierender Kante (Menge $S_i$) mindestens einer der beiden Vertices (Elemente von $S_i$) gewählt werden muss. In der Vorlesung (Foliensatz 10) wird ein Algorithmus vorgestellt, der \textsc{Vertex Cover} mit Lösungsgröße $k$ als Paramiter in $1.342^k \cdot n^{O(1)}$ löst. 

Nutzt man diesen Algorithmus für das Lösen von $3$-\textsc{Hitting Set}, kann der Disjoint-Algorithmus in $1.342^k \cdot n^c$ gelöst werden. Das Lemma aus der Vorlesung sagt, dass dann ein Compressions-Algorithmus für das Problem in $O((k+1)(1+1.342)^k \cdot n^c)$ gelöst werden kann. Der $(k+1)$-Faktor kann ignoriert werden, da $k+1$ in $d$ liegt und $d$ konstant ist. Damit ergibt sich eine Laufzeit von $2.342^k \cdot n^{O(1)}$.

%todo: Schritt zu (d-0.658)^k * n^O(n)

\how
%todo


\exercise{Matroide}

Ein transversaler Matroid ist ein Matroid, da er ein Unabhängigkeitssystem bildet und das Austauschargument gilt:

\begin{itemize}
\item Es gilt trivialer Weise $\emptyset \in L$ ist matchbar.

\item Gilt $A \in F$ und $B \subset A$, so ist auch $B \in F$. Ist $A$ Matchbar, so ist auch die Teilmenge $B$ matchbar. Das Matching in $B$ beinhaltet alle Knoten aus $L$, die auch in $A$ existieren. Jeder dieser Knoten kann mit der gleichen Kante, die es im Matching für $A$ auch hatte mit einem Knoten aus $R$ verbunden werden, also ist $B$ auch matchbar.

\item 
\end{itemize}

\subexercise
\subexercise

\exercise{Implementierung: \textsc{Feedback Vertex Set}}

\end{document}
