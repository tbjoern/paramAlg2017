%%%%%%%%%%%%%%%%%%%%%%%%%%%%%%%%%%%%%%%%%%%%%%%%%%%%%%%%%%%%%%%%%%%%%%
%% all the formatting stuff and packages

\documentclass[11pt]{article}

\parindent0em
\parskip.5em

\usepackage{xstring}

\usepackage{amsthm}
\usepackage[utf8]{inputenc}
\usepackage[ttscale=.85]{libertine}
\usepackage{libertinust1math}
% \usepackage[libertine,cmintegrals,cmbraces,vvarbb]{newtxmath}
\usepackage[T1]{fontenc}
\usepackage{microtype}
\usepackage{amsmath}
\usepackage{amssymb}
\usepackage{xspace}
\usepackage{ngerman}
\usepackage{graphicx}
\usepackage{lastpage}
\usepackage{ifthen}
\usepackage{fp}
\usepackage{hyperref}
\usepackage{icomma}
\usepackage{paralist}
\usepackage[ngerman,onelanguage,noend]{algorithm2e}
\DontPrintSemicolon

\makeatletter
\DeclareRobustCommand{\bfseries}{%
   \not@math@alphabet\bfseries\mathbf
   \fontseries\bfdefault\selectfont
   \boldmath
}
\makeatother

\usepackage[headsep=1cm]{geometry}

\usepackage{fancyhdr}

\fancypagestyle{plain}{%
  \renewcommand{\headrulewidth}{0pt}%
  \fancyhf{}
  \rhead{\includegraphics[width = 2.4cm]{fig/hpi_logo.pdf}}
  \lhead{\textbf{\sffamily Parametrisierte Algorithmen} \\
    \textbf{\sffamily Wintersemester 2017/2018} \\
    \sffamily Thomas Bläsius}
  \cfoot{\thepage}
  \rfoot{\ifthenelse{\thepage < \pageref{LastPage}}{\textit{bitte
        wenden}}{}}
}
\fancypagestyle{normal}{%
  \renewcommand{\headrulewidth}{0pt}%
  \fancyhf{}
  \cfoot{\thepage}
  \rfoot{\ifthenelse{\isodd{\thepage} \and \thepage <
      \pageref{LastPage}}{\textit{bitte wenden}}{}}
}

\usepackage{titlesec}

\titleformat{\section}%
[hang]%
{\Large\bfseries\sffamily}%
{Aufgabe \thesection:}%
{.5em}%
{}%
[]

\renewcommand{\thesubsection}{\alph{subsection}}
\usepackage{titlesec}
\titleformat{\subsection}%
[runin]%
{\bfseries\sffamily}%
{Teilaufgabe (\thesubsection)}%
{0pt}%
{}%
[]

\usepackage{titlesec}
\titleformat{\subsubsection}%
[hang]%
{\large\bfseries\sffamily}%
{\thesection}%
{.5em}%
{}%
[]

\titlespacing{\section}{0pt}{1.5ex}{.5ex}

\titlespacing{\subsection}{0pt}{.5ex}{.5em}

\titlespacing{\subsubsection}{0pt}{1ex}{.5ex}


%%%%%%%%%%%%%%%%%%%%%%%%%%%%%%%%%%%%%%%%%%%%%%%%%%%%%%%%%%%%%%%%%%%%%%
%% commands to use in the exercise sheet/solution

\newcommand{\sheet}[2]{ %
  \title{\textbf{\sffamily Übungsblatt #1}\\[-0.5ex]
    {\normalsize \sffamily Abgabe bis #2}}
  \date{}
  \maketitle
  \pagestyle{normal}
  \vspace{-2cm}
}

\newcommand{\solution}[2]{ %
  \title{\textbf{\sffamily Musterlösung zum Übungsblatt #1}\\[-0.5ex]
    {\normalsize \sffamily Erstellt von #2}}
  \date{}
  \maketitle
  \pagestyle{normal}
  \vspace{-2cm}
}

\newcommand{\exercise}[2][]{%
  \section{#2 \hfill {\normalsize#1}}%
}

\newcommand{\subexercise}{%
  \subsection{}%
}

\newcommand{\how}[1]{%
  \subsubsection*{Wie kommt man drauf?}%
}




\DeclareMathOperator{\vc}{vc}

\begin{document}

\solution{2}{Marvin Mirtschin, Tobias Stengel und Sören Tietböhl}

\exercise{Punkte und Geraden}

\exercise{Dominierende Mengen}

Wir gehen dabei ähnlich vor, wie beim FPT-Algorithmus für Independent Set aus der Vorlesung.

Wir nehmen an, dass die Aufgabenstellung nur wissen möchte, ob es ein Connected Dominating Set der Größe $k$ gibt. Weiterhin nehmen wir an, dass wir nur Zusammenhängende Graphen betrachten.
Diese Eigenschaft können wir leicht in $\Theta(m)$ mit einer Breitensuche sicherstellen.
Die Baumzerlegung ist als schöne Baumzerlegung vorliegend.

Der einzige Unterschied zum Algorithmus aus der Vorlesung besteht darin, wie wir die Regeln für Introduce, Forget und Join Knoten wählen.

Für Introduce Regeln gehen wir wie folgt vor (Beispiel von Folie 4, 3.Vorlesung):

\begin{center}
\begin{tabular}{c c c c c c c c}
    $\emptyset$ & $\{4\}$ & $\{5\}$ & $\{6\}$ & $\{4,5\}$ &$\{4,6\}$ &$\{5,6\}$ &$\{4,5,6\}$\\
    $\infty$ & 2 & 2 & $\infty$ & 3 & 3 & 3 & 4\\
\end{tabular}
\end{center}

wird zu

\begin{center}
\begin{tabular}{c c c c}
    $\emptyset$ & $\{4\}$ & $\{6\}$ & $\{4,6\}$\\
    $\infty$ & 2 & 3 & 3\\
\end{tabular}
\end{center}

wobei sich der Eintrag für 4 aus dem Minimum von $\{4\}$ und $\{4,5\}$,
für 6 aus dem Minimum von $\{6\}$ und $\{5,6\}$, sowie für $\{4,6\}$ aus dem Minimum von $\{4,6\}$ und $\{4,5,6\}$ berechnet.
Der Eintrag für $\emptyset$ kommt von $\{6\}$.

Für introduce Knoten gehen wir wie folgt vor:

\begin{center}
\begin{tabular}{c c c c}
    $\emptyset$ & $\{4\}$ & $\{6\}$ & $\{4,6\}$\\
    $\infty$ & 2 & 3 & 3\\
\end{tabular}
\end{center}

wird nach Hinzufügen von $\{7\}$ zu

\begin{center}
\begin{tabular}{c c c c c c c c}
    $\emptyset$ & $\{4\}$ & $\{7\}$ & $\{6\}$ & $\{4,7\}$ &$\{4,6\}$ &$\{7,6\}$ &$\{4,7,6\}$\\
    $\infty$ & 2 & $\infty$ & $\infty$ & 3 & 3 & $\infty$ & 4\\
\end{tabular}
\end{center}

Da 7 keine Verbindung zu 6 hat, kann also 6 oder 7 alleine nicht funktionieren. Auch $\{7,6\}$ kann nicht funktionieren, da 4 7 vom Rest separiert. ($\{4,6\}$ ist ein Seperator für 7)

Ein Sonderfall kann noch auftreten, dafür Betrachten wir den grünen Teil des Graphen auf Folien 8 und 9 , Vorlesung 3.
Beim Schritt von $\{U,n\}$ zu $\{U,5,n\}$ tritt der Fall ein, dass weder $U$ noch $n$ mit 5 verbunden sind. Nach der obigen Berechnung würde dann überall $\infty$ stehen.
Damit die Ergebnisse brauchbar bleiben, ignorieren wir hier die \emph{Connected}-Eigenschaft und berechnen nur ein Dominating Set. Wir tun also so, als ob 5 mit beiden anderen Knoten verbunden wäre und berechnen die Werte entsprechend.
Das Ganze funktioniert deshalb, weil 5 anschließend mit einem anderen Strang verjoined wird, welcher 5 mit einem der anderen Knoten verbinden kann.

Solch ein Join Knoten muss in der schönen Baumzerlegung immer existieren, da der Graph zusammenhängend ist.
Betrachten wir den Fall, dass ein Knoten $x$ introduced wird, der keine Verbindung zu den Knoten in seinem Bag $B = \{x, b_1, \dots , b_n\}$ hat.
Dann muss es einen Pfad $\{x, k_1, \dots, k_i, b_k\}$ geben, der $x$ mit einem der Knoten aus $B$ verbindet (weil der Graph zusammenhängend ist).

Die Knoten $\{k_1, \dots, k_i\}$ können bisher in keinem Unterknoten von $B$ aufgetaucht sein, da $B$ ein Seperator ist. Falls einer der Knoten aufgetaucht wäre, könnten nicht mehr alle Verbindungen durch Bags abgedeckt werden, da $x$ grade introduced wurde. (Und in einer schönen Baumzerlegung wird jeder Knoten auf dem Pfad zur Wurzel max 1 mal introduced)

D.h. es muss noch einen zweiten Ast des Baumes geben, auf dem dann die Knoten des Pfades abgearbeitet werden. Dort befinden sich dann auch $x$ und $b_k$ irgendwann in einem Bag. Damit die Teilbaum Eigenschaft erhalten bleibt müssen $x$ und $b_k$ also an einem Join beteiligt sein. Demnach werden $B$ und $B_2$ bzw darüberliegende Bags verjoined.

Bleibt also nur noch die Vorgehensweise eines Joins zu beschreiben. Das passiert identisch zum Algorithmus aus der Vorlesung.
D.h. der neue Wert einer Menge berechnet sich aus der Addition der alten Werte minus der Kardinalität.

\exercise{Baumweite planarer Graphen}

\end{document}