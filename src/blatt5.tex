%%%%%%%%%%%%%%%%%%%%%%%%%%%%%%%%%%%%%%%%%%%%%%%%%%%%%%%%%%%%%%%%%%%%%%
%% all the formatting stuff and packages

\documentclass[11pt]{article}

\parindent0em
\parskip.5em

\usepackage{xstring}

\usepackage{amsthm}
\usepackage[utf8]{inputenc}
\usepackage[ttscale=.85]{libertine}
\usepackage{libertinust1math}
% \usepackage[libertine,cmintegrals,cmbraces,vvarbb]{newtxmath}
\usepackage[T1]{fontenc}
\usepackage{microtype}
\usepackage{amsmath}
\usepackage{amssymb}
\usepackage{xspace}
\usepackage{ngerman}
\usepackage{graphicx}
\usepackage{lastpage}
\usepackage{ifthen}
\usepackage{fp}
\usepackage{hyperref}
\usepackage{icomma}
\usepackage{paralist}
\usepackage[ngerman,onelanguage,noend]{algorithm2e}
\DontPrintSemicolon

\makeatletter
\DeclareRobustCommand{\bfseries}{%
   \not@math@alphabet\bfseries\mathbf
   \fontseries\bfdefault\selectfont
   \boldmath
}
\makeatother

\usepackage[headsep=1cm]{geometry}

\usepackage{fancyhdr}

\fancypagestyle{plain}{%
  \renewcommand{\headrulewidth}{0pt}%
  \fancyhf{}
  \rhead{\includegraphics[width = 2.4cm]{fig/hpi_logo.pdf}}
  \lhead{\textbf{\sffamily Parametrisierte Algorithmen} \\
    \textbf{\sffamily Wintersemester 2017/2018} \\
    \sffamily Thomas Bläsius}
  \cfoot{\thepage}
  \rfoot{\ifthenelse{\thepage < \pageref{LastPage}}{\textit{bitte
        wenden}}{}}
}
\fancypagestyle{normal}{%
  \renewcommand{\headrulewidth}{0pt}%
  \fancyhf{}
  \cfoot{\thepage}
  \rfoot{\ifthenelse{\isodd{\thepage} \and \thepage <
      \pageref{LastPage}}{\textit{bitte wenden}}{}}
}

\usepackage{titlesec}

\titleformat{\section}%
[hang]%
{\Large\bfseries\sffamily}%
{Aufgabe \thesection:}%
{.5em}%
{}%
[]

\renewcommand{\thesubsection}{\alph{subsection}}
\usepackage{titlesec}
\titleformat{\subsection}%
[runin]%
{\bfseries\sffamily}%
{Teilaufgabe (\thesubsection)}%
{0pt}%
{}%
[]

\usepackage{titlesec}
\titleformat{\subsubsection}%
[hang]%
{\large\bfseries\sffamily}%
{\thesection}%
{.5em}%
{}%
[]

\titlespacing{\section}{0pt}{1.5ex}{.5ex}

\titlespacing{\subsection}{0pt}{.5ex}{.5em}

\titlespacing{\subsubsection}{0pt}{1ex}{.5ex}


%%%%%%%%%%%%%%%%%%%%%%%%%%%%%%%%%%%%%%%%%%%%%%%%%%%%%%%%%%%%%%%%%%%%%%
%% commands to use in the exercise sheet/solution

\newcommand{\sheet}[2]{ %
  \title{\textbf{\sffamily Übungsblatt #1}\\[-0.5ex]
    {\normalsize \sffamily Abgabe bis #2}}
  \date{}
  \maketitle
  \pagestyle{normal}
  \vspace{-2cm}
}

\newcommand{\solution}[2]{ %
  \title{\textbf{\sffamily Musterlösung zum Übungsblatt #1}\\[-0.5ex]
    {\normalsize \sffamily Erstellt von #2}}
  \date{}
  \maketitle
  \pagestyle{normal}
  \vspace{-2cm}
}

\newcommand{\exercise}[2][]{%
  \section{#2 \hfill {\normalsize#1}}%
}

\newcommand{\subexercise}{%
  \subsection{}%
}

\newcommand{\how}[1]{%
  \subsubsection*{Wie kommt man drauf?}%
}




\DeclareMathOperator{\vc}{vc}

\begin{document}

\solution{5}{Sören Tietböhl, Tobias Stengel}

\exercise{\textsc{Max Sat}}

\subexercise

Um Hall's Theorem anwenden zu können, muss ein bipartiter Graph \linebreak $G=(V_1 \cup V_2,E)$ vorliegen. $V_1$ ist dabei die Menge der Variablen aus $\varphi$ und $V_2$ die Menge der Klauseln aus $\varphi$. Sei $v_1 \in V_1$ und $v_2 \in V_2$. Es existiert eine Kante $e \in E$, wenn die Variable $v_1$ in der Klausel $v_2$ vorkommt (original oder negiert). Das heißt das Setzen von $v_1$ entscheidet, ob die Klausel $v_2$ erfüllt wird.

\textbf{Reduzierung der Variablenanzahl auf maximal $k$:} Angenommen es gilt $n \leq k$. Dann gibt es bereits weniger als $k$ Variablen und somit muss die Anzahl nicht weiter reduziert werden. 

Interessant ist also der Fall, wenn es noch mehr als $k$ Variablen gibt, also wenn gilt $|V_1| > k$. In dem bipartiten Graphen kann nun geprüft werden, ob es ein Matching gibt, welches alle Knoten von $V_1$ abdeckt. Gibt es ein solches Matching, gibt es auch mindestens $k$ Klauseln, die erfüllt werden können. Grund dafür ist, dass wir noch mehr als $k$ Variablen haben (sonst müssten wir nicht weiter reduzieren). Jede dieser Variablen kann so gesetzt werden, dass sie mindestens eine Klausel, nämlich ihren Matchingpartner, erfüllt. Also können $\geq n$ Klauseln erfüllt werden, die Instanz ist lösbar. 

Wird kein solches Matching gefunden, besagt Hall's Theorem, dass eine inklusionsminimale Menge $X\subset V_1$ gibt, sodass $|X| > |N(X)|$ gilt. Dann kann folgende Reduktionsregel angewandt werden:

\begin{itemize}
\item[Reduktionsregel 1:] Wähle ein beliebiges Element aus $x \in X$ und entferne $x$ aus $V_1$. Diese Regel ist sicher, da sie die Anzahl der erfüllbaren Klauseln nicht verändert. Da $X$ inklusionsminimal ist, wird jede der Klauseln, die mit durch das Setzen von $x$ erfüllt werden konnte, immernoch von anderen Variablen aus $X$ abgedeckt. Die Anzahl der erfüllbaren Klauseln bleibt also unverändert, die Variablenanzahl wird um 1 reduziert.
\end{itemize}

Mithilfe dieser Reduktionsregel, kann iterativ die Anzahl der Variablen auf maximal $k$ reduziert werden. Kann die Regel nicht mehr angewandt werden, obwohl $n>k$ gilt, gibt es im bipartiten Graph ein Matching, was bedeutet, dass die Instanz in jedem Fall lösbar ist.

\textbf{Reduzierung der Klauselzahl auf maximal $2k$:} Sei eine Variablenbelegung gegeben. Vereinigt man die Menge der mit dieser Belegung erfüllten Klauseln mit der Menge der von der komplementären Belegung erfüllten Klauseln, erhält man die Menge aller Klauseln. Bei $m$ Klauseln steht also fest, dass eine Variablenbelegung oder ihr Komplement mindestens $\frac{m}{2}$ Klauseln erfüllen. Ist $m > 2k$, so werden von einer Belegung oder ihrem Komplement mindestens $\frac{2k}{2}=k$ Klauseln erfüllt, die Instanz ist also lösbar. Gilt $m \leq 2k$, muss die Klauselanzahl nicht weiter reduziert werden.

\subexercise


\exercise{\textsc{3-Hitting Set}}

Damit eine Instanz lösbar ist, muss mindestens ein Element aus jeder Menge $S_i$ zu $H$ hinzugefügt worden sein. Man kann also alle Mengen $S_i$ durchgehen und effiziente Elemente zu $H$ hinzufügen. Gibt es mehrere Möglichkeiten Elemente zu $H$ hinzuzufügen, wird verzweigt und mehrere Lösungen werden weiterverfolgt. 

Beim \textsc{3-Hitting Set} gilt $|S_i| \leq 3$. Daher werden zunächst drei Fälle unterschieden:

\begin{itemize}
\item[Fall 1:]$\exists S_i: |S_i|=1$. Beinhaltet eine Menge $S_i$ nur ein Element, so muss dieses Element zu $H$ hinzugefügt werden, da sonst $H \cap S_i = \emptyset$ gilt. Das Element wird also zu $H$ hinzugefügt und es muss nur eine Lösung weiter verfolgt werden.

\item[Fall 2:]$\exists S_i: |S_i|=2$. Beinhaltet eine Menge $S_i$ zwei Elemente, werden zwei Lösungen weiterverfolgt. In jeder dieser Lösungen wird je eines der beiden Elemente zu $H$ hinzugefügt und das andere nicht. Anschließend werden die restlichen Mengen betrachtet, wobei je ein Element weniger berücksichtigt werden muss. Der Verzweigungsvektor für diesen Fall ist also $(1,1)$, die Basis damit $b=2$.

\item[Fall 3:]$\forall S_i: |S_i|=3$. Gibt es keine Menge mehr, die nur ein oder zwei Elemente enthält, enthalten alle übrigen Mengen drei Elemente. Nun kann man eine Mengen $S_x$ wählen und Schnittmengen mit anderen Mengen betrachten. Die Kardinalitäten der Schnittmengen führen zu folgender Fallunterscheidung

\begin{itemize}
\item[Fall 3.1:]$\exists S_y: |S_x \cap S_y| = 2$. Gibt es eine Menge, sodass die Schnittmenge mit $S_x$ die Kardinalität 2 hat, kann man entweder eines der beiden Elemente zu $H$ hinzufügen, oder die beiden Elemente zu $H$ hinzufügen, welche nicht in beiden Mengen vorkommen. Der Verzweigungsvektor ist also $(1,1,2)$, die Basis damit $b\approx 2,414$.

\item[Fall 3.2:]$\exists S_y: |S_x \cap S_y| = 1$. Gibt es eine Menge, sodass die Schnittmenge mit $S_x$ die Kardinalität 1 hat, kann man dieses zu $H$ hinzufügen oder hat vier Möglichkeiten je eines der anderen beiden Elemente der einen Menge mit einem der anderen beiden Elemente der anderen Menge zu kombinieren und zu $H$ hinzuzufügen. Der Verzweigungsvektor ist also $(1,2,2,2,2)$, die Basis damit $b\approx 2,562$.

\item[Fall 3.3:]$\forall S_y: |S_x \cap S_y| = 0$. Gibt es keine Menge, die ein Element mit $S_x$ gemeinsam hat, muss ein Element aus $S_x$ zu $H$ hinzugefügt werden. Da keines der Elemente in $S_x$ in einer anderen Menge vorkommt, kann einfach das erste Element gewählt werden. Es muss also nicht verzweigt werden und die Anzahl der zu betrachtenden Mengen verringert sich um 1.
\end{itemize}
\end{itemize}

In jedem Schritt ist die Basis $b \leq 2,562$. Außerdem werden in jedem Schritt Elemente zu $H$ hinzugefügt, sodass mindestens eine weitere Menge $S_i$ abgedeckt wird. Also sind $\leq n$ Schritte notwendig, um alle Mengen abzuarbeiten. Der Algorithmus liegt also in $2,562^k \cdot n^{O(1)}$.

\exercise{\textsc{Vertex Cover} mit beschränkten Suchbäumen}

Implementiert wurden die  fünf in der Vorlesung beschriebenen Regeln zur Verzweigung der Lösungsfindung.

In \textit{vertex\_cover} werden zunächst der minimale und maximale Grad des aktuellen Graphen ermittelt, um anhand dieser Werte zu entscheiden, welche Regel als nächstes angewandt wird. Zunächst wird geprüft, ob Regel 1 angewandt werden kann, dann nacheinander Regel 2, 3 und 4 bzw. 5. Kann eine Regel angewandt werden, wird dies getan und die mögliche Anwendung anderer Regeln vernachlässigt. Regel 4 und 5 aus den Folien wurden zusammengefasst, da jeweils gleich verzweigt wird: einmal wird der Knoten zum Vertex-Cover hinzugefügt und einmal alle seine Nachbarn. 

Innerhalb der Regeln werden zunächst zusammenhänge zwischen den Nachbarn des aktuellen Knotens ermittelt (zum Beispiel, ob diese durch eine Kante verbunden sind), um so den genauen Unterfall zu ermitteln. Dann werden gemäß der Folien (wenn nötig) mehrere Teilberechnungen gestartet, indem mehrere Aufrufe von \textit{vertex\_cover} mit unterschiedlichen Restgraphen und bisherigen Vertex-Covern aufgerufen. Je nachdem Welche Teillösung das kleinere Vertex-Cover zurückgibt, wird dieses Vertex-Cover von der aktuellen Regel zurückgegeben.

Einige Regeln wurden auf kleinen Graphen getestet, wo sie das richtige Ergebnis lieferten. Bei komplexeren Graphen lief das Programm nicht durch, weshalb auch keine Lösungen für die Graphen abgegeben wurden. Die Vermutung ist, dass das Löschen von Knoten aus dem Graphen nicht korrekt funktioniert, da es in späteren Regelanwendungen noch Knoten/Kanten gibt, die schon im Vertex-Cover sind und damit aus dem Graphen gelöscht worden sollten.


\end{document}
